\skiptooddpage
\section[A lineáris programozás dualitás tétele]{A lineáris programozás dualitás
  tétele (két alakban), a lineáris programozás alapfeladatának bonyolultsága}

\subsection{A lineáris programozás dualitás tétele (két alakban)}
A tétel kijelenti, hogy ha:

$ \begin{rcases}
		\mbox{LP: max}\left\{cx:Ax \leq b\right\} \\
		\mbox{megoldható}                         \\
		\mbox{felülről korlátos}
	\end{rcases} \Rightarrow$ $\begin{cases}
		\circled{1} \begin{cases}
			            \mbox{DP: min} \left\{ yb:yA = c, y \geq 0 \right\} \\
			            \mbox{megoldható}                                   \\
			            \mbox{alulról korlátos}
		            \end{cases} \\
		\circled{2} \begin{cases}
			            \mbox{LP--nek } \exists \mbox{ maximuma} \\
			            \mbox{DP--nek } \exists \mbox{ minimuma}
		            \end{cases}            \\
		\circled{3} \mbox{ maximum } = \mbox{ minimum}
	\end{cases}$

Ezzel ekvivalens alakja az LP és a DP-nek a:
\begin{align*}
	\mbox{LP: max} & \left\{ cx:Ax \leq b, x \geq 0 \right\}  \\
	\mbox{DP: min} & \left\{ yb:yA \geq c, y \geq 0 \right\}.
\end{align*}

Az \circled{1}--est bizonyítottuk az LP korlátosságánál, a \circled{2} és
\circled{3}--hoz felhasználjuk a következő \emph{lemmát}:

\[
	\begin{rcases}
		Ax \leq b \mbox{ megoldható} \\
		t \in \mathbb{R}             \\
		Ax \leq b \mbox{--nek } \not \exists~x \mbox{ megoldása, hogy } cx \geq t
	\end{rcases} \Rightarrow
	\parbox[t]{6cm}{
		$\{yA=c, y \geq 0\}$--nak $\exists$ olyan megoldása amelyre  $yb < t$.}
\] A lemma \emph{bizonyításhoz}, átfogalmazzuk a bal oldalt mint egy Farkas
lemma alak:
$\begin{cases} Ax \leq b \\
		-cx \leq -t\end{cases}$, és tudjuk, hogy a rendszer nem megoldható (nem
létezik $x$ ami teljesiti). Alkalmazzuk rá a Farkas--lemmát, ennek második
alakja felírható, mint ($\lambda \geq 0, y
	\geq 0$):
\begin{align*}
	\begin{bmatrix} y & \lambda\end{bmatrix}
	\begin{bmatrix} A \\ -c \end{bmatrix} = yA - \lambda c = 0 & \Rightarrow
	yA = \lambda c,                                                          \\
	\begin{bmatrix} y & \lambda\end{bmatrix}
	\begin{bmatrix} b \\ -t \end{bmatrix} =yb~ - \lambda t < 0 & \Rightarrow
	yb < \lambda t
\end{align*}

Ha $\lambda=0 \Rightarrow \begin{rcases}
		yA = 0,   \\
		y \geq 0, \\
		yb <  0\end{rcases} \xRightarrow{\text{Farkas--lemma}}$ $Ax \leq b$ nem
megoldható, de ez ellentmondás az állításnak, tehát $\lambda$ nem lehet nulla
($\lambda \neq 0$). Legyen $y'=\frac{1}{\lambda}y$ így az egyenletrendszer
$y'A=c,~y'\geq0,~y'b<t$ alakú lesz és ezzel megadtunk minden $y$--ra egy
$y'$--et amely teljesíti a lemma kijelentését.

A \circled{2} \emph{bizonyításához} tegyük fel, hogy $\not \exists$ maximum
LP--n, ekkor legyen $t=\mbox{sup}\left\{cx:Ax \leq b \right\}$ (hiszen bármely
$x \subseteq R, x \neq \emptyset, x$ felülről korlátosnak $\exists$ szuprémuma,
legkisebb felső korlátja, ha még maximuma nincs is). Ha $t$ nem maximum
$\Rightarrow Ax \leq b$--nek $\not \exists~cx \geq t$--t teljesítő megoldása.
Ekkor:

\[ cx = (yA)\cdot x = y (Ax) \leq yb < t.\]

Így $yb$ egy $t$-nél kisebb felső korlát $cx$-re, de ez ellentmondás, tehát t
nem szuprémum és feltevésünk hamis volt, létezik maximum LP--n. A DP--n történő
bizonyításhoz használjuk fel az LP bizonyítást, átírva a feladatott max--ra ($E$
-- egység mátrix):

\[ max \left\{ (-b)^T y^T :
	\begin{cases}
		(~~A^T)y^T & \leq~~c^T \\
		(-A^T) y^T & \leq -c^T \\
		(-E~~)y^T  & \leq~~0   \\
	\end{cases}  \right\} \]

A \circled{3} \emph{bizonyításához} már láttuk, hogy a \[ \mbox{max}\left\{
	cx:Ax \leq b \right\} \leq
	\mbox{min}\left\{ yb:yA = c, y \geq 0 \right\}
\] fennáll. Legyen $t=\mbox{min}\{ yb:yA~=~c, y~\geq~0 \}$ és tegyük fel, hogy a
fenti egyenlőtlenségben nem egyenlőség áll.  Tehát ekkor az LP--nek
$\not\exists~cx \geq t$ kifejezést teljesítő megoldása, amiből következik,
hogy a duális feladatnak van olyan megoldása amire $yb<t$. De ez ellentmondás t
választásának, tehát feltevésünk hamis volt.

\subsection{LP bonyolultság}

Az LP feladat megfogalmazható, mint eldöntési probléma: $\exists?$ az $Ax \leq
	b$--t kielégítő x vektor amelyre $cx\geq t$? A probléma NP--beli. Tanú egy ilyen
$x$. Ugyanakkor coNP--beli is, a duális feladat megoldása tanú erre. $x, y$
mérete véges.

\begin{description}
	\item[1947 -- Dantzig -- Szimplex módszer] Hatékony gyakorlati feladatokon, de
	      exponenciális komplexitású.
	\item[1979 --  Hacsijan -- ellipszoid módszer] Polinomiális komplexitás, de
	      gyakorlatban lassú.
	\item[1984 --  Karmakar -- belső pontos módszer]  Polinomiális, hatékonyabb az
	      ellipszoidnál, de ez is a gyakorlatban lassabb a szimplexnél.
\end{description}
