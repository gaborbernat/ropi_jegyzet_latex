\section{Teljesen polinomiális approximációs séma a részösszeg problémára}

Egy probléma polinomiális approximációs sémával közelíthető, ha
$\forall~\epsilon>0$--ra van rá $(1+\epsilon)$-approximáció. Ez nem mindig elég,
mert ha a közelítő algoritmus lépésszáma például $2^{\frac{1}{\epsilon}}n^4$,
még mindig exponenciálisan hosszú ideig fut. Ilyenkor ugyanis rögzített
$\epsilon$--ra polinomiális az algoritmus, tehát a feltételnek megfelel.

Egy probléma \emph{teljesen} polinomiális approximációs sémával közelíthető, ha
$\forall~\epsilon>0$--ra van rá $(1+\epsilon)$-approximáció, ami
$\frac{1}{\epsilon}$--ban is polinomiális. Például a metrikus utazó ügynök
problémára nincs P approximációs séma, de az euklideszi utazó ügynök problémára
van teljesen P approximációs séma.

\subsection{Részösszeg probléma}

\[
\begin{rcases}
A=\{a_1, a_2, \cdots a_n\} \\
a_{i=\overline{1,n}},~t \in F
\end{rcases} \Rightarrow \exists ?~B \subseteq A,  \mbox{ hogy } \sum b_i =t
\]

Partíciós problémáról beszélünk, ha $t = \frac{\sum a_i}{2}$. Mindkét esetben a
feladat NP--teljes. A részösszeg optimizálási probléma, ha azon $B$--re
vagyunk kíváncsiak amelyre $\sum b_i$ maximális és $\sum b_i \leq t$ . A feladat
NP nehéz, mert a részösszeg probléma visszavezethető rá: ha találunk olyan B
halmazt, amire $\sum bi = t$, akkor igen a válasz a részösszeg problémára,
különben nem. A feladat nem NP-beli, mert nem eldöntési probléma, tehát nem is
NP-teljes.

\vspace{0.4cm}
\emph{A részösszeg probléma probléma teljesen polinomiális sémával közelíthető.}
 \vspace{0.4cm}

A bizonyítás egy konkrét algoritmus lesz:

\begin{enumerate}
  \item \emph{Pontos megoldást adó algoritmus}\\
Ez az algoritmus gyakorlatilag nem más, mint egy „brute-force” megoldó, amely
minden lehetséges részösszeget kiszámol. Legyen $a_1 \leq a_2 \leq \cdots \leq
a_n$. Definiáljunk két halmazsorozatot:
\begin{align*}
L_i' = \{ I + a_i | I \in L_i\}, & \mbox{ ahol } L_0 = \{ 0\} \\
L_{i+1}=L_i \cup L_i' & \\
\end{align*}

Az optimális részletösszeg max$\{I|I \in L_n \mbox{ és } I \leq t\}$ lesz.

Például, ha $a_1=3~a_2=5~a_3=7$:
{
\begin{align*}
    &&~L_0 &= \{ 0 \} \\
L_0'&= \{ 3 \} &~ L_1 &= \{ 0, 3\} \\
L_1'&= \{ 5,8 \} &~ L_2 &= \{ 0, 3, 5, 8\} \\
L_2'&= \{ 7, 10, 12, 15 \} &~ L_3 &= \{ 0, 3, 5, 7, 8, 10, 12, 15\} \\
\end{align*}}
  \item \emph{Polinomiális közelítő algoritmus} \\
  Egyrészt vegyük észre, hogy az $L_i'$ halmazokból nincs szükségünk a $t$--nél
  nagyobb elemekre. Töröljük ki ezeket: $L_i' = (L_i+a_i) \cap [0 \cdots t]$.

  Ugyanakkor, ritkítsuk a halmazt $\delta$--val: legyen $L$ a növekvő sorrendbe
  rendezett halmaz. Ezután a halmaz minden elemét vizsgáljuk meg, növekvő
  sorrendben, ha létezik kisebb elem ($m$) az aktuálisnál ($l$) amelyre
  $l<m(1+\delta)$ (az aktuális $1+\delta$--szor nagyobb) akkor dobjuk ki az
  aktuálist ($l$--et). A ritkítás után bármely két szomszédos elem hányadosa
  legalább $1+\delta$. A ritkított halmaz mérete felülről becsülhető:
  $|L_{\mbox{ritkított}}| \leq \log_{1+\delta}{t+2}$.
  \item Ha a fenti algoritmus során a halmazok $t$--nél nagyobb elemeit minden
  lépésben levágjuk és az eredményt $\delta = \frac{\epsilon}{2n}$--vel
  ritkítjuk, $(1+\epsilon)$--approximációt kapunk a problémára, és a lépésszám
  $n$--ben, $\log t$--ben és $\frac{1}{\epsilon}$--ban is polinomiális lesz.
\end{enumerate}

Az $1+\epsilon$ approximációs tulajdonságot lássuk be. Ehhez elősző figyeljük
meg a következő lemmát:
\[ \ln{(1+X)} \geq \frac{X}{1+X}, \mbox{ ha } X \geq 0.\]

Ha $X=0$ mindkét oldal nulla lesz, tehát az állítás igaz. Ha $X \geq 0$:

\[ \left( \ln{(1+X)} - \frac{X}{1+X}\right)' = \frac{1}{1+X} - \frac{(1+X)-X}{(1+X)^2} =
\frac{1}{1+X} - \frac{1}{(1+X)^2} > 0
\]
Tehát a derivált pozitív, amiből következik, hogy a függvény monotonan nő, tehát
a lemma igaz.

Most az algoritmus lépészámát vizsgálva, a kapott halmaz mérete:

\[ \log_{1+\delta}{t}=\log_{1+\frac{\epsilon}{2n} }{t} = \frac{\ln{t}}{\ln(1+\frac{\epsilon}{2n})}
\leq \frac{\ln{t}}{\frac{\frac{\epsilon}{2n}}{1+\frac{\epsilon}{2n}}}
= \left(1 + \frac{\epsilon}{2n} \right) \cdot 2n \cdot \frac{\ln{t}}{\epsilon} \]

Az egyenlőtlenséghez felhasználtuk a lemmát $\left(\ln(1+\dfrac{\epsilon}{2n})
\geq \dfrac{\dfrac{\epsilon}{2n}}{1+\dfrac{\epsilon}{2n}}\right)$.
Megfigyelhetjük, hogy a lépésszám mind $n$--ben, $\log t$--ben és
$\frac{1}{\epsilon}$ is polinomiális.

Az algoritmus során $n$ darab lista összefésülését kell elvégezni, aminek a
komplexitása O$(\sum |L_i|)=$O$(n \cdot |L_i|)$. Felhasználva a korábbi
becslésünket az algoritmus $n$--ben négyzetes, $\log{t}$--ben lineáris, azaz
mindkét esetben polinomiális. Most olyan megoldást keressünk, amely az
approximációs séma követelményeit teljesíti, azaz megoldása $\geq
\dfrac{\mbox{OPT}}{1+\epsilon}$.

A kapott válasz relatív hibájának kis értékére való bizonyításához vegyük
figyelembe, hogy $L_i$ lista ritkításakor legfeljebb $\frac{\epsilon}{n}$
nagyságú hibát okozunk a megmaradó képviselő érték és a ritkítás előtti érték
között. $i$--re vonatkozó teljes indukcióval bizonyítható, hogy ezzel
eredményünk az optimum $(1+ \frac{\epsilon}{2n})^n$ hányada lesz. Ez meg teljes
appróximáció lesz mert:

\[ \underbrace{\frac{\mbox{OPT}}{1+\epsilon}}_{\mbox{teljes approximáció}} \leq
\underbrace{\frac{\mbox{OPT}}{(1+ \frac{\epsilon}{2n})^n}}_{\mbox{amennyit max
rontunk}} \leq \mbox{ kapott érték } \leq \mbox{ OPT. }\]

Ahhoz, hogy ez igaz legyen bizonyitanunk kell, hogy $(1+\frac{\epsilon}{2n})^n
\leq 1 + \epsilon$ is igaz:

\begin{align*}
\left(1+\frac{\epsilon}{2n}\right)^n &=
\sum_{i=0}^{n}\left(\frac{\epsilon}{2n}\right)^i \cdot 1^{n-i} \cdot \binom{n}{i} =
 \sum_{i=0}^{n}\left(\frac{\epsilon}{2n}\right)^i \cdot \frac{n!}{i!(n-i)!}
 \leq 1 + \sum_{i=0}^{n}\left(\frac{\epsilon}{2n}\right)^i \cdot n^i \\
 &\leq 1 + \sum_{i=1}^{n}\left(\frac{\epsilon}{2}\right)^i = 1 + \frac{\epsilon}{2}
 + \frac{\epsilon}{4} + \cdots + \frac{\epsilon}{2^n} \leq 1 + \epsilon
\end{align*}

A bizonyítás során felhasználásra került a binomiális tétel és a mértani sor
összegképlete.
