\skiptooddpage 
\section{\texorpdfstring{$k$--polimatroid}
		 {k--polimatroid}}

Egy  $f:2^E \mapsto \mathbb{N}$ egy $k$--polimatroid rangfüggvénye, ha teljesül:

\[
\begin{cases}
\mbox{KP1})& r(\emptyset)=0\\
\mbox{KP2})& r(\{X\}) \leq k, \mbox{ ekvivalens alak } r(X) \leq k |X|  \\
\mbox{KP3})& X \subseteq Y \Rightarrow r(X) \leq r(Y)\\
\mbox{KP4})& r(X \cup Y) + r (X \cap Y) \leq r(X) + r(Y) \Leftrightarrow r(X \cup Y) \leq r(X) + k|Y| \\
\end{cases}
\]

Ha $r(\{X\}) \leq 1$, tehát $k=1$ akkor $r$ egy matroid rangfüggvénye. 

\subsection{\texorpdfstring{$k$--polimatroid matching probléma}
		 {k--polimatroid matching probléma}}


Ha $r$ egy $k$--polimatroid rangfüggvény és $X\subseteq E$--re $r(X)=k|X|$,
akkor $X$--et egy $k$--matching halmaznak nevezzük. A $k$--polimatroid matching
probléma felírható mint:

\[
\begin{rcases}
E \mbox{ alap halmaz} \\
r \mbox{ egy } k \mbox{--polimatroid rangfüggvény} \\
p \in \mathbb{N} \\
\end{rcases}  \xRightarrow {?\exists X \subseteq E}
\begin{cases}
r(X)=k|X| \mbox{, tehát X } k \mbox{--matching} \\
|X| \geq p 
\end{cases}
\]

Figyeljük meg pár speciális esetét a problémának:

\begin{itemize}
  \item Adott egy $G$ gráf és egy $t \in \mathbb{N}$. $\nu(G) \geq t$? \footnote{
   $\nu(G)$ a $G$ gráfban található független élek maximális száma.
  } A probléma $2$--polimatroid matchingként: $r(X)=|X$
  által lefedett pontok halmaza$| \leq 2|X|$.
  \item $\begin{rcases}
  \mbox{két matroid} \\
  t \in \mathbb{N}
  \end{rcases}$ létezik  e $X \subseteq E$: $\begin{cases}
  |X| \geq t \\
  X \in F_1 \cap F_2~(\mbox{matroid metszet probléma})
  \end{cases}$ 
  $2$--polimatroid matchingént megfogalmazva: $r(X)=r_1(X)+r_2(X) \leq 2|X|$
  \item A két korábbi problémát összevonva $\nu(G)$ egy adott $G$ páros gráfban?
  
  Hogy megfogalmazzuk $k$--polimatroid problémának felhasználjuk a fenti két
  feladatot. A páros gráf két csúcshalmaza legyen $A$ és $B$. $M_1$ grafikus
  matroidban két él párhuzamos ha létezik közös pontjuk $A$ ponthalmazban.
  $M_2$ hasonlóan van definiálva $B$--re. 
\end{itemize}

\subsection{Bonyolultság}
Teljes általánosságban a probléma nem oldható meg polinom időben. Az általánosság
a függvény megadási módjára vonatkozik. $\forall x \subseteq E$--re egységnyi
idő alatt $r(X)$ értéke megtudható, de ettől eltekintve a $2$ polimatroid
rangfüggvényéről semmit sem tudunk.

Ennek bizonyításához induljunk ki egy $|E|=2n$ alaphalmazból és ennek egy
$|E_0|=n$ részhalmazából. Továbbá legyen az alábbi két polimatroid rangfüggvény:

\[
\begin{cases}
\forall X \subseteq E \mbox{ és } |X|<n, & r_1(X)=r_2(X)=2|X| \\
\forall X \subseteq E \mbox{ és } |X|>n, & r_1(X)=r_2(X)=2n \\
\forall X \subseteq E \mbox{ és } |X|=n \mbox{ és } X = E_0, & r_1(X)=r_2(X)=2n-1\\
\forall X \subseteq E \mbox{ és } |X|=n \mbox{ és } X \neq E_0, & r_1(X)=2n-1 \mbox{ és }r_2(X)=2n\\
\end{cases}
\]

Mindkét függvény $2$--polimatroid (a szubmodularitás tulajdonság bizonyítását
nem részletezzük). Legyen most $p=n$. Arra a kérdésre, hogy van-e legalább
p-elemű $2$-matching, a válasz nyílván nemleges lesz az $r_1$ és igenlő az $r_2$
input függvény esetén. 

Ha egy algoritmus teljes általánosságban meg tudja oldani a matroidpárosítási
problémát, akkor a legrosszabb esetben mind $\dbinom{2n}{n}$ darab $n$ elemű n
elemű részhalmazra meg kell kérdeznie az $r_i$ függvény értékét.

Hiszen, ha csak egy híján mindegyikre kérdezné meg és mindig $2n-1$ választ
kapna, akkor még nem tudhatná, hogy melyik esetünkről van szó ($r_1$ vagy
$r_2$). Ez pedig már $n$ függvényében exponenciálisan sok lépés (az analízisből
ismert, hogy $\dbinom{2n}{n}$ aszimptotikusan egyenlő $\frac{4^n}{\sqrt{\pi
n}}$--nel).

\subsection{Lovász--tétel}

Vegyünk egy $\mathbb{M}_{k \times 2n}$ mátrixot amely $\mathbb{R}$ felett van
koordinátázva. Oszlopai legyenek rendre $\{a_1, b_1, \cdots, a_n, b_n \}$ majd
definiáljunk az $I=\{1,2,\cdots,n\}$ index halmazon egy $r$ függvényt. $r(X)$
értéke egy adott $X\subseteq I$ legyen az $\cup_{i \in X}\{a_i, b_i\}$
vektorhalmaz által kifeszített altér dimenziója (hány független elem van a
mátrixból kiválasztott oszlopok közül). Ilyenkor $r$ egy $2$--polimatroid
rangfüggvény ($\forall~i$--re $r(\{a_i, b_i\}=2$ -- nincs hurokél és $a_i$
párhuzamos $b_i$--vel).

\vspace{0.4cm}
\emph{
A matroidpárosítási probléma polinom időben megoldható, ha a 2-polimatroid
rangfüggvény egy adott valós elemű M mátrixból a fent leírt módon nyerhető.}
