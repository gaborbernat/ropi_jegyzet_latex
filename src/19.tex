\newpage
\section{Ütemezési algoritmusok}

Az ütemezési feladatt megfogalmazható mint:
\[
\begin{rcases}
J_i \mbox{ -- munkák} \\
P_i \mbox{ -- megmunkálási idő} \\
w_i \mbox{ -- súly} \\
d_i \mbox{ -- határidő} \\
r_i \mbox{ -- rendelkezésre álló idő} \\
C_i \mbox{ -- befejezési idő} \\
\end{rcases}
\Rightarrow 
\parbox[t]{9cm}{ $+$ egy adott időben egy gépen egy munka folyik \\
és létezik egy célfügvény amire nézve optimális ütemezést keresünk.}
\]

Egy halmazfeladat gyors leírására a következő struktúrt használjuk:
\[
\underbrace{\alpha}_{\mbox{gépi infó}} | \underbrace{\beta}_{\mbox{ütemezési infó}
} | \underbrace{\gamma}_{\mbox{optimalizálási kritérium}}
\]

\begin{description}
  \item[gépi infórmáció] alatt a rendelkezésre álló gép számosságot értjük. Lehet:
  \begin{itemize}
  \item $1$ -- egy gép áll rendelkezésre.
  \item $P_m$ -- $m$ darab párhuzamosan futó gép
  \item $P$ -- a párhuzamosan futó gépek száma nincs rögzitve, tartalmazza az $1,2,\cdots$ feladatokat is.
\end{itemize}
  \item[ütemezési infórmáció] azt mondja meg, hogy milyen ütemezést illető
  megkötéseink vannak a halmaz elemeire:
  \begin{itemize}
  \item \emph{prec} -- létezzik egy írányitott aciklikus gráf amely
  meghatározza, hogy a munka elkezdéséhez mely munkák kellet már befejeződjenek.
  \item $r_j$ -- egy--egy munkának tudjuk, hogy mely időpontól ál rendelkezésünkre.
  \item $P_j$ -- tudjuk, hogy egy--egy munka elvégzése mennyi időtvesz igénybe
  (ez mindig adott).
\end{itemize}
  \item[optimalizálási kritérium] arra vonatkozik, hogy a célfüggvényben mire
  fektetünk hangsúlyt:
  \begin{itemize}
  \item $C_{\mbox{max}}=\mbox{max}(C_j)$ -- utolsó munka befejezési ideje legyen minnél
  kisebb (gyorsan befejezni a csomagot).
  \item $\sum C_j = \sum \frac{C_j}{n}$ -- a munka átlagos befejezési ideje
  legyen a lehető legkisebb (átlagosan kevesett várjunk a gépekre).
\end{itemize}
\end{description} 

\subsection{ \texorpdfstring {$ 1||C_{max} $} {1||Cmax} }

Egy gépre teljes átfutási idő szerint optimizálunk. $C_{\mbox{max}}=\sum_{i=1}^{n}
P_i \Rightarrow$ tehát csak felrakjuk a munkákat sorba, ügyelve arra, hogy ne 
létezzen pillanat amikor a gép áll és optimális megoldást kapunk.

\subsection{ \texorpdfstring {$ 1|prec|C_{max} $} {1|prec|Cmax} }

Most olyan sorrendben végezzük el a felrakást, hogy ne legyen munka amire a
függőség még nem fejezödőt be. Ez a sorrend egy irányitott aciklikus gráfban
(DAG) nem mást mint a topologikus sorrend. Tehát első pontnak vegyünk egy forrás
pontot a precedencia gráfból, ezt töröljük és ismételjűk e folyamatott amig
létezzik pont a gráfban. Ez polinomiális időben optimális megoldást ad.

\subsubsection{Topologikus sorrend meghatározása}

Adjunk hozzá a bemeneti DAG--hoz egy extra $s$ élet amelyet a végén törölni 
fogunk. Minden $s \mapsto v, v \in V$ élt rakjuk be a gráfba.