
\newpage
\section{Elhagyás és összehúzás. Matroidok direkt összege, összefüggősége. T test felett
reprezentálható matroid duálisának T feletti reprezentálhatósága.}

Legyen $M=(E,F)$ matroidon $X \subseteq E$ halmazra definiáljuk a következő
műveleteket:
\begin{description}
  \item[elhagyás] az $M \setminus X=(E-X, F \setminus Y)$, ahol $F \backslash
  Y=\{Y\subseteq E-X, Y \in F\}$ (tehát az új matroidot úgy kapjuk, hogy
  alaphalmazból elhagyjuk az $X$--beli elemekt és a halmazrendszerből meg azon
  elemeket amelybe ezek részt vesznek).
  \item[összehuzás]  az $M / X=(E-X, F / X)$, ahol az $M/X$ rangfüggvénye felírható
  $r(Y)=r(X \cup Y) - r(X), Y \in  E-X$ alakban (ha kivesszük az alaphalmazból az
  összehuzott elemeket az új alaphalmaz -- $Y$ -- rangja egyenlő a teljes kezdeti alaphalmaz 
  -- $X \cup Y = E$-- és kivett alaphalmaz különbségével).
\end{description}

Az elagyások és az összehuzás műveletek prioritása azonos, tehát felcserélhető. $M$ matroid 
\emph{minora} annak egy elhagyás és összehuzás sorozata. Bármely matroid minora előáll
$N=(M \setminus A) / B$ alakban, ahol $A$ és $B$ diszjunkt halmazok.

\emph{Az elhagyás és az összehuzás duális művelet: $\begin{cases}
M^*/X=(M \setminus X)^*, \\
M^* \setminus X = (M / X)^*.
\end{cases}$}

A bizonyitáshoz elég az elsőt belátni, mert ha az igaz, mindkét oldal duálisát
véve és $M$ helyet $M^*$--ra alkalmazva azt következik a második kijelentés.
Tudjuk tehát, hogy az duális matroid összehuzásának ($M^*/X$) rangfüggvénye
(legyen ez $r_1$):

\begin{align*}
r_1(Y) &=\underbrace{r^*(X \cup Y)}_{|X \cup Y| + r(E-X-Y) - r(E)} - 
		 \underbrace{r^*(X)}_{|X| + r(E-X) - r(E)} \\
	   &= |Y| + r(E-X-Y) - r(E-X) 
\end{align*}

Ugyanakkor a matroidból X elhagyásával ($M^*/X$) a rangfüggvény ($r_2$):

\[ r_2(Y) = |Y| + r(T-Y) - r(T),\] ahol $T = E-X$ az $M\setminus X$ matroid
alaphalmazza. Látjuk, hogy $r_1(Y)=r_2(Y)\Rightarrow$ minden $Y$--ra a a két
matroid megegyezik, s ezzel bizonyitásunk teljes.

\begin{description}
  \item[direkt összeg] $\begin{rcases}
  M_1=(E_1,F_1) \mbox{ matroid}, \\
  M_2=(E_2,F_2) \mbox{ matroid}, \\
  E_1, E_2 \mbox{ diszjunkt, nem üres} \end{rcases} N = M_1+ M_2$ alaphalmaza
  $E_1 \cup E_2$, és $X \subseteq E_1 \cup E_2$ halmaz akkor független a direkt
  összegben, ha $X \cap E_1$ független $M_1$--ben és $X \cap E_2$ független
  $M_2$--ben. 
  \item[egy matroid \emph{összefüggő}] ha nem áll elő matroidok direkt
  összegeként. A grafikus matroid akkor összefüggő, ha a gráf kétszeresen
  összefüggő.
  \item[$M=(E,F)$ reprezentálható] T test felett, ha bármely elem az alaphalmazból
  $T$ feletti vektor. 
  \item[$M$ matroid koordinázható]T test felett, ha létezik olyan mátrix, amelynek oszlopai
  $T$ felett vektorok, és az ezek által meghatározott lináris matroid izomor $M$--el. 
\end{description}

Bármely $r=r(E), n=|E|$ matroidra létezzik $A \in \mathbb{M}_{r\times n}$ mátrix
amelyel leírható $M=(E,F)$ matroid (és a mátrix sorai lineárisan függetlenek).
Az $A$ mátrix megalkotásához $r$ sorra van szükség, ha a matroid több elemet
tartalmaz válaszunk ki $r$ lineárisan függetlent. Ekkor a jobb oldali alakra
hozható $A$ mátrix, ettől még ugyanazt a matroidot koordinázza:

\[
\overbrace{
\begin{array}{|lcr|}
\hline
1 & \cdots & r\\
\vdots  & \mbox{det} \neq 0 &\\
r &  & \\
\hline
\end{array}}^{\mbox{egység alaká alakít}}
\cdot~
\begin{array}{|lcr|}
\hline
1 & \cdots & n\\
\vdots  & A &\\
r &  & \\
\hline
\end{array}
=
\begin{array}{|lcr|}
\hline
1 & \cdots & n\\
\vdots  & B &\\
r &  & \\
\hline
\end{array}
=
\begin{array}{|lcr|lcr|}
\hline
1 & \cdots & r &1&\cdots&n-r\\
\vdots  & I_r &&\vdots&A'&\\
r &  & & r&&\\
\hline
\end{array}
\]

\emph{Ha $M=(E,F)$ matroid reprezentálható $T$ test felett akkor a duálisa ($M*$) is.}

A bizonyitáshoz hozzuk a matroid mátrixát oly alakra, hogy a mátrix bal oldalán
egy egységmátrix alakuljon ki. Legyen ennek $r$ sorra, ekkor $E_r$ egységmátrix
M egy bázisa, míg $A_0$ oszlopai a többi elemek. Megprobáljuk belátni, hogy $A'=(-A_0^T|E_{n-r})$
is reprezentálja a duális matroidot.

\colorlet{ColorGreyish}{black!10}
\[ 
M=
\begin{array}{|cc|c|c|rc|}
\hline
1       &  0     & \multicolumn{1}{>{\columncolor{lightgray}}l}{\color{black}0} & \multicolumn{1}{>{\columncolor{gray}}l}{\color{white}C} &     &\\
\vdots &  1     & \multicolumn{1}{>{\columncolor{lightgray}}l}{\color{black}0} &  \multicolumn{1}{>{\columncolor{gray}}l}{\color{white}} & A_0 &\\
0       &  \cdots  & \multicolumn{1}{>{\columncolor{ColorGreyish}}l}{\color{black}1} &\multicolumn{1}{>{\columncolor{lightgray}}l}{\color{black}}   &     &\\
\hline
\end{array}
\Rightarrow
M^*=
\begin{array}{|c|c|cc|lcr|}
\hline
\multicolumn{1}{>{\columncolor{gray}}l}{\color{white}}        & & 1&  &\multicolumn{1}{>{\columncolor{lightgray}}l}{\color{black}0} & \multicolumn{2}{>{\columncolor{lightgray}}l}{\color{black}}  \\
\multicolumn{1}{>{\columncolor{gray}}l}{\color{white}-C^T}    & &  & 1&\multicolumn{2}{>{\columncolor{lightgray}}l}{\color{black}} &\multicolumn{1}{>{\columncolor{lightgray}}l}{\color{black}0}  \\
\cline{1-1} \cline{5-7}
\multicolumn{1}{>{\columncolor{lightgray}}l}{\color{black}}   & &  &  & \multicolumn{1}{>{\columncolor{ColorGreyish}}l}{\color{black}1} & \multicolumn{1}{>{\columncolor{ColorGreyish}}l}{\color{black}} &\multicolumn{1}{>{\columncolor{ColorGreyish}}l}{\color{black}0}  \\
\multicolumn{1}{>{\columncolor{lightgray}}l}{\color{black}A_0^T}     & &  && \multicolumn{1}{>{\columncolor{ColorGreyish}}l}{\color{black}}  & \multicolumn{1}{>{\columncolor{ColorGreyish}}l}{\color{black}1}   & \multicolumn{1}{>{\columncolor{ColorGreyish}}l}{\color{black}} \\
\multicolumn{1}{>{\columncolor{lightgray}}l}{\color{black}} & &  &  & \multicolumn{1}{>{\columncolor{ColorGreyish}}l}{\color{black}0}  & \multicolumn{1}{>{\columncolor{ColorGreyish}}l}{\color{black}}  & \multicolumn{1}{>{\columncolor{ColorGreyish}}l}{\color{black}1} \\ \hline
\multicolumn{1}{c}{\mathsmaller{r-t}} & \multicolumn{1}{c}{\mathsmaller{t}} & \multicolumn{2}{c}{\mathsmaller{r-t}} & \multicolumn{3}{c}{\mathsmaller{n-2r+t}}
\end{array} 
\]

$M$ matroid és a duálisának a mátrixa természetes modón megfelelhetőek
egymásnak, az ábrán balról jobbra az oszlopcsoportok megfelelnek egymásnak.
Válaszunk ki az $M$ valamely bázisának megfelelő részmátrixot $A$--ban.
Feltehető, hogy ennek első $t$ oszlopa éppen $E_r$ utolsó $t$ oszlopa, míg a
maradék $r-t$ oszlop $A_0$ első $r-t$ oszlopa. Ennek a mátrixnak determinánsa
akkor nem nulla, ha $C$ determinánsa nem nulla. Mivel $B$ bázis, ezért $C$ nem
szinguláris.

$B^*=E-B$ elemeinek $A'$--ben az a mátrix felel meg, amely első $r-t$ oszlopa
$-A_0^T$, a többi pedig $E_{n-r}$ utolsó $n-2r+t$ oszlopa. Ennek a felső
blokk--háromszög mátrix determinánsa akkor nem nulla, ha $-C^T$ determinánsa nem
nulla.  Azaz, hogy $C$ nem szinguláris. Tehát a bázis komplementere is bázis, és
fordítva, és ezért $A'$ valóban a duálist reprezentálja.
