\newpage
\section{$k$-matroidpartíciós probléma ($\mathrm{MPP}_k$)}

\newcommand{\M}{\mathcal{M}}
\newcommand{\F}{\mathcal{F}}
\theoremstyle{definition}
\newtheorem*{definicio}{Definíció} \newtheorem*{tetel}{Tétel}
\newtheorem*{allitas}{Állítás} \newtheorem*{algoritmus}{Algoritmus}
\newtheorem*{pelda}{Példa}

$\mathrm{MPP}_k$ P-beli probléma.

input: $k$ matroid: $\M_i = (E, \F_i), i=1,\ldots,k$

kérdés: (az alábbi két kérdés ekvivalens)

\begin{itemize}
  \item $\bigvee_{i=1}^k \M_i = (E, 2^E)$
  \item $E = E_1 \cupdot \ldots \cupdot E_k$ úgy, hogy $\forall i E_i \in
  \F_i$-re. Feltehetõ, hogy az $E_i$ halmazok diszjunktak, ezért hívják a
  feladatot matroid partíciós problémának.
\end{itemize}

output:

\begin{figure}[hb]
\centering
%\includegraphics[width=120mm,keepaspectratio]{figures/M1M2.pdf}
\caption{Példa, hogy a mohó algoritmus miért nem mûködik: $E_1 = \{2\}$,
$E_2 = \{3\}$ állapotban nem tudunk újabb élt belevenni egyik halmazba sem. 
Jó partícionálás: $E_1 = \{1\}$, $E_2 = \{2, 3\}$.}
\label{fig:M1M2}
\end{figure}

\begin{itemize}
  \item ,,igen'' és a partíció
  \item ,,nem'' és $X \subseteq E$ úgy, hogy $|X|; \sum_{i=1}^k r_i(X)$. (Ez
  a tanú bizonyítja, hogy nincs jó partíció)
\end{itemize}

\begin{algoritmus}
Az algoritmus kezdetén $\forall i: E_i = \emptyset$ ($E_1, \ldots, E_n$
halmazok)

Minden lépésben készítünk egy $G'(V', E')$ segédgráfot. $V' = E \cup \{p_1, p_2,
\ldots, p_k\}$. $|E|=n$ esetén $|V|=n+k$.

A gráf élei: $(\overrightarrow{xy}) \in E' \Leftrightarrow$

\begin{itemize}
  \item vagy $x \in E, y = p_i, x \not\in E_i, E_i \cup \{x\} \in \F_i$ (az ábra
  felsõ részen a $p_i$-kbe mutató élek)\\
  Egy ilyen él azt jelenti, hogy $x$-et hozzávehetjük $E_i$-hez.
  \item vagy $x,y \in E, x \not\in E_i, y \in E_i$, $E_i \cup \{x\} \not\in
  \F_i, E_i \cup \{x\} - \{y\} \in \F_i$ (az ábra alsó részén lévõ élek)\\ Egy
  ilyen él azt jelenti, hogy $E_i$-hez hozzávehetjük $x$-et, ha $y$-t elhagyjuk
  belöle.
\end{itemize}

\begin{figure}[hb]
\centering
%\includegraphics[width=30mm,keepaspectratio]{figures/futas1.pdf}
\caption{}
\label{fig:futas1}
\end{figure}

Ha létezik irányított út $E-(\bigcup_i E_i)$-bõl $\{p_1, p_2, \ldots p_k\}$-ba,
akkor ez meghatároz egy módosítás-sorozatot, hogy melyik $E_i$-t hogyan kell
módosítanunk. Az út utolsó éle egy $p_j$-be megy, a többi $E$-beli elemek
között, vagyis összesen 1-gyel növeljük $E_i$-k össz. elemszámát. Ha egy {\it
legrövidebb} ilyen út mentén javítunk, akkor bizonyítható, hogy $E_i$-k a
módosítások után függetlenek maradnak (mi nem bizonyítjuk). Ezután az algoritmus
folytatódik, újra megcsináljuk a segédgráfot stb.

\end{algoritmus}

A futás eredménye:

\begin{itemize}
  \item ,,igen'' válasz: ha bevettünk minden pontot, az algoritmus megadja a partíciókat.
  \item ,,nem'' válasz: ha elakadtunk valahol.
\end{itemize}

,,Nem'' válasz esetén bizonyíték, hogy nincs jó partíció: $X$ az $E - \bigcup_i
E_i$-bõl elérhetõ pontok halmaza.

\[
|X| \sum_{i=1}^k |X \cap E_i| = \sum_{i=1}^k r_i(X \cap E_i) = \sum_{i=1}^k r_i(X)
\]

$\forall i$-re $r_i(X \cap E_i) = r_i(X)$, mert: $r_i(X) \geq r_i(X \cap E_i)$
triviális, $r_i(X) \leq r_i(X \cap E_i)$ pedig azért igaz, mert ha ha $r_i(X)
> r_i(X \cap E_i)$ lenne, akkor $(F_3)$ miatt $X \cap E_i$ kiegészíthetõ
lenne egy $u \in X - E_i$ elemmel úgy, hogy $\F_i$-ben továbbra is független
legyen: $(X \cap E_i) \cup \{u\} \in \F_i$, ezért:

\begin{itemize}
  \item vagy $E_i \cup \{u\} \in \F_i$, ekkor $u \rightarrow p_i$ él be lenne
  húzva a gráfba és mivel $u$ elérhetõ $E - \bigcup_i E_i$-bõl ($X$ definíciója
  miatt), ezért $p_i$ is, tehát nem álltunk volna meg, ellentmondás.
  \item vagy $E_i \cup \{u\} \not\in \F_i$, akkor van benne kör, ez a kör csak
  olyan lehet, ami $E_i-X$-be belemetsz (mivel ha $X$-ben lenne az egész kör,
  akkor $(E_i \cup \{u\})\cap X = (X \cap E_i) \cup \{u\}$ is összefüggõ lenne,
  de nem az). Legyen a körnek egy $E_i-X$-beli eleme $v$, ekkor $E_i \cup \{u\}
  - \{v\}$ független (mert független + 1 elemû halmazban csak egy kör lehet, ha
  ezt megszüntetjük, újra független lesz), tehát lenne $u \rightarrow v$ él a
  gráfban. Ez ellentmondás, mert akkor $v$ $X$-beli lenne.
\end{itemize}

Mindkét eset ellentmondás, tehát $r_i(X) = r_i(X \cap E_i)$, így $X$ tényleg jó
tanú arra, hogy nincs jó partíció.

\begin{figure}[hbt]
\centering
%\includegraphics[width=75mm,keepaspectratio]{figures/futas2.pdf}
\caption{}
\label{fig:futas2}
\end{figure}