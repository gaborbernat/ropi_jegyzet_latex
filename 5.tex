\skiptooddpage
\section[Egészértékű programozás]{Egészértékű programozás: a feladat bonyolultsága, korlátozás és szétválasztás (Branch and Bound)}

\subsection{Egészértékű programozás}
A feladat:
\begin{align*}
IP &: \left\{\mbox{max }{cx: Ax \leq b, x \mbox{ egész}}\right\} \\
DIP &: \left\{ \mbox{min }{yb: yA=c, y \geq 0, y \mbox{ egész}}\right\}
\end{align*}
\[ \mbox{max(IP)} \leq \mbox{max(LP)}= \mbox{min(DLP)} \leq \mbox{min(DIP)} \]

A feladat bonyolultságának meghatározásához fogalmazzuk át eldöntési problémára:
$\exists?~Ax \leq b,~ x$ egész között olyan vektor, amelyre $cx \geq t?$ Az IP
feladat NP--teljes. NP--beli mert x tanú. A dualitás tétel nem alkalmazható
mivel max(IP)~$<$~min(DIP) fennállhat, tehát a probléma nem coNP--beli.
\emph{NP teljes}, mivel visszavezethető a $3$--SAT problémára, és a maximális
független ponthalmazra (elég az egyik is).

\subsubsection{Maximális független ponthalmaz}

$ \begin{cases}
input: \mbox{G(v,e) gráf, h szám} \\
output: \alpha (G) >= h? \\
	\mbox{Minden csúcsra: }
	\forall x_i 0 <= x_i <= 1, x_i \in \mathbb{Z}
	\xRightarrow{}
	x_i = \begin{cases}
		\mbox{1, ha az i-edik csúcs bekerül} \\
		\mbox{0, ha nem}
	\end{cases} \\
\mbox{Minden élre: }
x_i + x_j <= 1 \\
max: x_1 + x_2 \ldots + x_n
\end{cases}$

Ezzel az IP segítségével megoldottuk a max. ftl. problémát, tehát az NP-teljes.

\subsubsection{$3$--SAT}
A bizonyításhoz megadjuk a visszavezetést: definiálunk egy eljárást, amely tetszőleges
logikai függvényhez megad egy olyan IP problémát, melyre akkor és csak akkor igenlő a
válasz, ha a függvény kielégíthető. A függvény alakja:

\[ f(x_1, x_2, \cdots, x_k) = \bigwedge_{i=1}^{d}{\left( x_{i_1}^{e_{i_1}} \vee 
x_{i_2}^{e_{i_2}} \vee x_{i_3}^{e_{i_3}}\right).} 
\]
Az IP probléma ismeretlenjei legyenek $z_{i=\overline{1,k}}$ egyenlőtlenségek,
ahol $0 \leq z_{i=\overline{1,k}} \leq 1$. Továbbá minden diszjunkcióra is felírunk egy
egyenlőtlenséget:


\begin{align*}
z_{i_1} &+ z_{i_2} &+      z_{i_3}  &\geq 1 & \mbox{ ha }~~x_{i_1} \vee~~x_{i_2}\vee~~x_{i_3}& \\
z_{i_1} &+ z_{i_2} &+ (1 - z_{i_3}) &\geq 1 & \mbox{ ha }~~x_{i_1} \vee~~x_{i_2} \vee \neg x_{i_3}& \\
z_{i_1} &+ (1-z_{i_2}) &+ (1 - z_{i_3})  &\geq 1 & \mbox{ ha }~~x_{i_1} \vee \neg x_{i_2} \vee\neg x_{i_3}&\\
(1-z_{i_1}) &+ (1-z_{i_2}) &+ (1 - z_{i_3})  &\geq 1 & \mbox{ ha } \neg x_{i_1} \vee \neg x_{i_2} \vee \neg x_{i_3}&
\end{align*}

Ha $f$ kielégíthető, akkor $z_i=1 \Leftrightarrow x_i=$ ,,igaz''. Ez lesz az
egyenletrendszer egészértékű megoldása. 

Megfordítva, ha az egyenlőtlenség--rendszert egész $z_i$ értékkel ki tudjuk
elégíteni akkor -- mivel minden egész vagy $0$, vagy $1$ -- az $x_i$=,,igaz''
$\Leftrightarrow z_i=1$ választással $f$--et kielégítő értéket adtunk a logikai
változónak.

Eközben a $\sum_{i=1}^{k}{z_i}$ összeg értétek valahol $0$ és $k$ között adódik,
tehát $c$--enk $(1,1,\cdots,1)$ választással tulajdonképpen a \emph{,,Van-e a fenti
egyenlőtlenségrenszerek feltételeit kielégítő (egész) vektorok között olyan,
melyre $cz \geq 0$?''} kérdést tesszük fel. Az így kapott IP feladatra a válasz
akkor és csak akkor igenlő, ha az f kielégíthető.

\subsection{A korlátozás és szétválasztás az IP feladatra}

Alakja $\mbox{max} \left\{cx:Ax \leq b, f \leq x \leq g| x,f,g \in \mathbb{Z}^n
\right\},$ ahol az $f$ és $g$ korlátok, amelyek biztosítják, hogy a feladat véges sok
lépésben megoldható. A metódus kulcsgondolata, hogy az IP feladat megoldásához először
megoldjuk mint LP, ha az eredmény egész megvagyunk, egyébként tovább bontjuk két kisebb
IP feladatra. Legyen $w^*=cx^*$, az egyenlőtlenség mindig $Ax \leq b$ marad.

Az algoritmus folyamata:

\begin{enumerate}
  \setcounter{enumi}{-1}
  \item $\overbrace{L=\left\{ \left(
  \overbrace{f}^{\text{alsó}},\overbrace{g}^{\text{felső}},\overbrace{\infty}^{w
  \text{ -- maximum értéke}} \right) \right\}}^{\text{IP részfeladatok}},
  \overbrace{w^*=-\infty}^{\text{eddigi legjobb célfüggvény érték}},
  \overbrace{x^*}^{\text{legjobb megoldás}}$
  \item Ha $L=\emptyset \Rightarrow$ \emph{vége} és a megoldás az aktuális $w^*$
  és $x^*$.
  
  Egyébként $L$--ből vegyük ki IP$_i$ feladatot $\Rightarrow (f_i, g_i, w_i)$.
  \item Ha $w_i \leq w^*$, IP$_i$--nek a megoldása nem lehet $w^*$--nál jobb,
  folytassuk az első ponttól (ez a \emph{Bound} lépés).
  \item Oldjuk meg a relaxált IP$_i$ feladatot (LP$_i$--t).
  \\ Ha $\not \exists$ megoldás visszalépünk az első ponthoz.
  \\ Egyébként: $w^{LP}_{i}$ -- maximum érték,  $x_i$ -- maximum hely.
  \item Ha $w^{LP}_{i} \leq w^*$, IP$_i$ feladat és rész feladatai maximum értéke is
  legfeljebb $w^*$, lépjünk vissza az első ponthoz.
  \item Ha $w^{LP}_{i} > w^* \mbox{ és } x_i \in \mathbb{Z}^{n}$ akkor $ w^* = w^{LP}_{i}, x^* =
  x_i$ (adat frissítés) és visszalépünk az első ponthoz.
  \item Választunk egy közbenső értéket $(f_{i_j} \leq t \leq g_{i_j})$ és egy
  ezt meghatározó $x_j$ elágazás változót $x_i$--ből ($j$ egy pozíció $x$
  vektorban, $i$ az IP feladat sorszámát jelöli). Ezután $L$--ben elhelyezzünk
  két új feladatott:  $(t+1, g, w^{LP}_{i})$ és $(f, t, w^{LP}_{i})$. Lépjünk vissza az első
  ponthoz (\emph{Branch} lépés).
\end{enumerate} 

Az algoritmus véges sok lépésben leáll és megtalálja a feladat optimumát. A
véges sok lépést $f$ és $g$ vektor garantálja, hiszen a kezdetben kitűzött
feladatnak csak véges sok rész problémája lehet a felhasznált korlátok miatt. Az
optimális részt indirekt bizonyítjuk: legyen $w_0$ az optimum, de az eljárás
$w^*<w_0$--t kapta.

Állítjuk, hogy $L$--ben mindig kellet, hogy legyen olyan IP$_i$ feladat amelynek
az optimum értéke $w_0$. Ez kezdetben fennáll, és az algoritmus egy ilyen IP
feladattal a $5.$--ik vagy $6.$--ik lépéshez jut el. Ha az ötödikbe jut akkor az
algoritmus mégis csak megtalálja a $w_0$ optimumot, ami ellentmondás.

Ha a hatodik lépés következik be, akkor a két keletkező IP feladat közül az
egyikre teljesül, hogy a vizsgált tartományában a $w_0$--hoz tartozó optimum
benne van, így $w_0$ optimumot mindenképp megtaláljuk. Tehát már csak az lehet,
hogy az eljárás során a lista sosem ürült ki, de ekkor az algoritmus sose
állhatott volna le, ami a keresett ellentmondás.

\subsection{Gyakorlati tanácsok}
\begin{itemize}
  \item LIFO alapú új feladat választás $L$--ből, mert a megoldás várhatóan
  mélyen van a fába és ez tud legmélyebbre leggyorsabban lenyúlni. Ugyanakkor a
  duál szimplex módszerrel lehetséges így a korábbi számitások újra felhasználása.
  \item Elágazásnál a legkevésbé egész $x_j$--t válasszuk elágazási változónak
  (ez a relaxált IP feladatt megoldásának helyvektora, azaz azt az indexet ahol
  e vektor elemének törtrésze $\frac{1}{2}$-höz legközelebb van), közbülső
  értéknek pedig ennek egész értékét (lefelé kerekitve) válasszuk. Ez azt
  jelenti, hogy az $f$ és a $g$ vektorban csak ez az indexen levő értéket
  modósitsuk,a többi feltétel marad a korábbi.
  
  \[
  f=\begin{bmatrix}
  1 \\
  2 \\
  3 
  \end{bmatrix},
  g = \begin{bmatrix}
  4 \\
  5 \\
  6 
  \end{bmatrix},
  x_0 = \begin{bmatrix}
  1.2 \\
  3.6 \\
  5.7 
  \end{bmatrix}
  \Rightarrow
  f_b=\begin{bmatrix}
  1 \\
  2 \\
  3 
  \end{bmatrix},
  g_b = \begin{bmatrix}
  4 \\
  3 \\
  6 
  \end{bmatrix},
  f_j=\begin{bmatrix}
  1 \\
  4 \\
  3 
  \end{bmatrix},
  g_j = \begin{bmatrix}
  4 \\
  5 \\
  6 
  \end{bmatrix}
  \] A fenti példában $x_0$ a talált optimum helyvektora. A $0.5$-höz
  legközelebb a $3.6$ van, ez a második változó indexet jelenti, és értéke lekerekitve $3$. A
  többi megkötés ($f$ és $g$ vektor elemei) marad a korábbi. Tehát $t=3$ és $f$
  és $g$ vektort ennek megfelelően módosítjúk $t$ és $t+1$ értékekkel. 
\end{itemize}